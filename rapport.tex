\documentclass[a4paper,10.5pt]{article}
\usepackage[margin=2.5cm]{geometry}
\linespread{1}

\usepackage[T1]{fontenc}
\usepackage[utf8]{inputenc}
\usepackage[swedish]{babel}

% \usepackage{fontspec}
% \setmainfont{Linux Libertine O}

\usepackage{listings}
\usepackage{amsmath}
\usepackage{amsfonts}
\usepackage{lmodern}
\usepackage{pbox}
\usepackage{fancyhdr}
\pagestyle{fancy}
\renewcommand{\headrulewidth}{ 1.0pt }
\renewcommand{\footrulewidth}{ 0.4pt }
\fancyhead{} % clear all headers
\fancyfoot{} % clear all footers

% E:even page, O:odd page, L:left, C:center, R:right
%\fancyhead[LE,RO]{\thepage}
%\fancyfoot[C]{Albert Einstein, 2009}

\fancyhead[L]{DD1350 HT15 Laboration 1}
\fancyhead[R]{Mikael Forsberg, Robin Gunning}
\fancyfoot[C]{\thepage}

\lstset{
    basicstyle=\footnotesize\ttfamily,
    breaklines=true,
    xleftmargin=0.6cm,
    showstringspaces=false,
    frame=none,
    numbers=left,
    language=Java,
    inputencoding=utf8,
    extendedchars=true,
    literate={≤}{{$\leq$}}1
}

\newcommand{\Tp}[1]{\mathrm{Tp}(#1)}
\newcommand{\Ln}[1]{\mathrm{Ln}(#1)}
\newcommand{\St}[1]{\mathrm{St}(#1)}
\newcommand{\Mt}[1]{\mathrm{Mt}(#1)}
\newcommand{\Rs}[1]{\mathrm{Rs}(#1)}
\newcommand{\Rd}[1]{\mathrm{Rd}(#1)}

\newcommand{\predrow}[3]{
    \pbox[t]{20cm}{\texttt{#1}} & \pbox[t]{20cm}{\texttt{#2}\\ \raisebox{-0.1cm}{\pbox[t]{20cm}{#3}} \vspace{0.33cm}}\\ 
}

\begin{document}
\pagenumbering{gobble}
\title{Laboration 1\\ Beviskontroll med Prolog\\ \vspace{0.2cm} \small Kungliga Tekniska Högskolan\\ Logik för dataloger\\ DD1350, HT2015}
\author{Mikael Forsberg <miforsb@kth.se>, Robin Gunning <rgunning@kth.se>}
\maketitle

\newpage
\tableofcontents

\newpage
\pagenumbering{arabic}
\section{Inledning}
Denna laboration gick ut på att implementera ett program för automatisk
kontroll av satslogiska bevis med naturlig deduktion i programspråket
Prolog.

\section{Metod}
I detta avsnitt presenteras den algoritm som utformades för att
utföra beviskontroller, och som sedan implementerades i Prolog.

\subsection{Algoritm - informell beskrivning}
Vår algoritm arbetar sig igenom beviset uppifrån och ned. Varje
rad innehåller en regelmotivation. Algoritmen validerar för varje
rad att den hänvisade regeln tillämpats korrekt på de rader som
eventuellt hänvisats till. Boxar hanteras med rekursion, genom att 
mer eller mindre se på innehållet som ett eget bevis, som dock kan
hänvisa till gällande formler utanför boxen (sådana formler måste
finnas ovanför boxen).

\subsection{Algoritm - formell beskrivning}
\textbf{Indata}: en lista med premisser $P$, en slutsats $S$ och ett
bevis $B$ där $B$ utgörs av en lista där varje element $E_i, i = 0,1,2,...,n$ antingen
är en trippel på formen (\emph{radnummer}, \emph{påstående},
\emph{motivering}) eller en lista innehållande fler element (en box).
\bigskip

\noindent
\texttt{<Triple> ::= tuple(lineno, statement, motivation)}

\noindent
\texttt{<P> ::= [] | cons(statement, <P>)}

\noindent
\texttt{<S> ::= statement}

\noindent
\texttt{<B> ::= cons(<Triple>, <B>) | cons(<B>, <B>) | cons(<Triple>, [])}

\bigskip
\noindent
I grammatiken ovan skall \texttt{cons} tolkas som att vara definierad så
att det går att lägga en lista inuti en annan lista, dvs. \texttt{cons(<List>, <List>)}.

% backus naur tar vi sen
% https://en.wikipedia.org/wiki/Backus%E2%80%93Naur_Form

\bigskip
\noindent
\textbf{Definitioner}:
\begin{itemize}
\item $\Tp{n}$ : den trippel $X \in B$ för vilken $\Ln{X} = n$
\item $\Ln{X}$ : radnumret i en trippel $X$
\item $\St{X}$ : påståendet i en trippel $X$
\item $\St{n}$ : påståendet $\St{\Tp{n}}$
\item $\Mt{X}$ : motiveringen i en trippel $X$
\item $\Rs{X}$ : de radnummer på samma boxnivå som hänvisas till i $\Mt{X}$
\item $\Rd{X}$ : de radnummer på djupare boxnivå som hänvisas till i $\Mt{X}$
\end{itemize}

\bigskip
\noindent
För samtliga definitioner ovan gäller att beviset $B$ finns tillgängligt implicit.
Symbolen $n$ skall tolkas som ett heltal. Den överlagrade funktionen $\St{...}$ skall tolkas verka på tripplar för argument
med stor bokstav, och heltal för argument med liten bokstav.

\bigskip
\noindent
\textbf{Att validera ett bevis $B$} givet en lista med premisser $P$ samt en slutsats $S$:
\begin{itemize}
\item Markera samtliga radnummer som stängda
\item Markera samtliga radnummer i tripplar på första nivån i $B$ som öppna.
\item Validera det första elementet $E_0 \in B$
\item Kontrollera att det sista elementet $E_n \in B$ är en trippel och att $\mathrm{St}(E_n) = S$.
\end{itemize}

\bigskip
\noindent
\textbf{Att validera ett element $X$} givet ett bevis $B$ samt en lista med premisser $P$:
\begin{itemize}
\item[] \textbf{OM} $X$ är en trippel
    \begin{itemize}
    \item[] \textbf{OM} $\Ln{X}$ är stängt
    \item[] $\qquad$ misslyckas
    \item[] \textbf{OM} något radnummer $\in \Rs{X}$ är stängt
    \item[] $\qquad$ misslyckas
    \item[] \textbf{OM} något radnummer $\in \Rs{X}$ är större eller lika med $\Ln{X}$
    \item[] $\qquad$ misslyckas
    \item[] \textbf{OM} något radnummer $\in \Rd{X}$ är öppet
    \item[] $\qquad$ misslyckas
    \item[] \textbf{VÄLJ} $\Mt{X}$ ur:
        \begin{itemize}
        \item[] premiss:
            \begin{itemize}
            \item[] \textbf{OM} $\St{X} \notin P$
            \item[] $\qquad$ misslyckas
            \end{itemize}
        \item[] copy(k):
            \begin{itemize}
            \item[] \textbf{OM} $\St{X} \neq \St{k}$
            \item[] $\qquad$ misslyckas
            \end{itemize}
        \item[] negationseliminering(k,m):
            \begin{itemize}
            \item[] \textbf{OM} $\St{k} \neq \lnot \St{m}$ \textbf{ELLER} $\St{X} \neq \bot$
            \item[] $\qquad$ misslyckas
            \end{itemize}
        \item[] falsumeliminering(k):
            \begin{itemize}
            \item[] \textbf{OM} $\St{k} \neq \bot$
            \item[] $\qquad$ misslyckas
            \end{itemize}
        \item[] icke-ickeintroduktion(k):
            \begin{itemize}
            \item[] \textbf{OM} $\St{X} \neq \lnot \lnot \St{k}$
            \item[] $\qquad$ misslyckas
            \end{itemize}
        \item[] LEM:
            \begin{itemize}
            \item[] \textbf{OM INTE} $\St{X}$ är på formen $p \lor \lnot p$
            \item[] $\qquad$ misslyckas
            \end{itemize}
        \item[] MT(k,m):
            \begin{itemize}
            \item[] \textbf{OM} $\St{k} \neq (p \rightarrow q)$ \textbf{ELLER} $\St{m} \neq \lnot q$ \textbf{ELLER} $\St{X} \neq \lnot q$
            \item[] $\qquad$ misslyckas
            \end{itemize}
        \item[] implikationseliminering(k,m):
            \begin{itemize}
            \item[] \textbf{OM} $\St{m} \neq (\St{k} \rightarrow \St{X})$
            \item[] $\qquad$ misslyckas
            \end{itemize}
        \item[] och-eliminering-1(k):
            \begin{itemize}
            \item[] \textbf{OM INTE} $\St{k}$ är på formen $\St{X} \land q$
            \item[] $\qquad$ misslyckas
            \end{itemize}
        \item[] och-eliminering-2(k):
            \begin{itemize}
            \item[] \textbf{OM INTE} $\St{k}$ är på formen $p \land \St{X}$
            \item[] $\qquad$ misslyckas
            \end{itemize}
        \item[] och-introduktion(k,m):
            \begin{itemize}
            \item[] \textbf{OM} $\St{X} \neq \St{k} \land \St{m}$
            \item[] $\qquad$ misslyckas
            \end{itemize}
        \item[] icke-icke-eliminering(k):
            \begin{itemize}
            \item[] \textbf{OM} $\St{k} \neq \lnot \lnot \St{X}$
            \item[] $\qquad$ misslyckas
            \end{itemize}
        \item[] antagande:
            \begin{itemize}
            \item[] \textbf{OM INTE} $\Ln{X}$ öppnar en box med djup $>0$
            \item[] $\qquad$ misslyckas
            \end{itemize}
\newpage
        \item[] eller-eliminering(k,m,p,q,r):
            \begin{itemize}
            \item[] \textbf{OM INTE} raderna $m$ och $p$ öppnar respektive stänger en och samma box 
            \item[] $\qquad$ misslyckas
            \item[] \textbf{OM INTE} raderna $q$ och $r$ öppnar respektive stänger en och samma box 
            \item[] $\qquad$ misslyckas
            \item[] \textbf{OM} $\St{k} \neq (\St{m} \lor \St{q})$
            \item[] $\qquad$ misslyckas
            \item[] \textbf{OM} $\St{p} \neq \St{r}$ \textbf{ELLER} $\St{X} \neq \St{p}$
            \item[] $\qquad$ misslyckas
            \end{itemize}
        \item[] pbc(k,m):
            \begin{itemize}
            \item[] \textbf{OM INTE} raderna $k$ och $m$ öppnar respektive stänger senast stängda\footnote{Den box som, sett från det radnummer man står på, senast stängdes.} box
            \item[] $\qquad$ misslyckas
            \item[] \textbf{OM} $\St{k} \neq \lnot \St{X}$ \textbf{ELLER} $\St{m} \neq \bot$
            \item[] $\qquad$ misslyckas
            \end{itemize}
        \item[] implikationsintroduktion(k,m):
            \begin{itemize}
            \item[] \textbf{OM INTE} raderna $k$ och $m$ öppnar respektive stänger senast stängda\footnote{Se fotnot 1} box
            \item[] $\qquad$ misslyckas
            \item[] \textbf{OM} $\St{X} \neq (\St{k} \rightarrow \St{m}$)
            \item[] $\qquad$ misslyckas
            \end{itemize}
        \item[] negationsintroduktion(k,m):
            \begin{itemize}
            \item[] \textbf{OM INTE} raderna $k$ och $m$ öppnar respektive stänger senast stängda\footnote{Se fotnot 1} box
            \item[] $\qquad$ misslyckas
            \item[] \textbf{OM} $\St{m} \neq \bot$ \textbf{ELLER} $\St{X} \neq \lnot \St{k}$
            \item[] $\qquad$ misslyckas
            \end{itemize}
        \end{itemize}
    \end{itemize}
\item[] \textbf{ANNARS} // $X$ är inte en trippel
    \begin{itemize}
    \item[] \textbf{OM INTE} $X$ är en lista där element 0 är en trippel motiverad med ''antagande''
    \item[] $\qquad$ misslyckas
    \item[] Markera samtliga rader på djup 0 i $X$ som öppna
    \item[] Validera $X$
    \item[] Markera samtliga rader på djup 0 i $X$ som stängda
    \end{itemize}
\item[] \textbf{OM} $\Tp{\Ln{X} + 1}$ inte är tomma listan
    \begin{itemize}
    \item[] Validera $\Tp{\Ln{X} + 1}$
    \end{itemize}
\item[] \textbf{ANNARS}
    \begin{itemize}
    \item[] färdigt, korrekt!
    \end{itemize}
 % ANNARS 
   % Validera $E_{n+1}$
\end{itemize}

% element_list ::= [] | [ element | element_list ]
% element ::= (linenr, statement, motivation) | element_list
% proof ::= element_list

% att validera B:

% för varje påstående i P, notera att P är sant

% för varje rad i B (varje trippel oavsett djup):
    % om djupet är > 1, markera raden som stängd

% för varje rad i B:
    % om raden är öppen:
        % om motiveringen är premiss
            % kontrollera att påståendet är noterat som sant
        % om motiveringen är en regel tillhörande naturlig deduktion
            % kontrollera att eventuella boxhänvisningar innehåller par av start- och slutradnr
             % som tillhör samma box
            % kontrollera att samtliga radnr i samtliga hänvisade boxar är stängda
            % kontrollera att regeln tillämpats korrekt på eventuella hänvisade radnr samt
             % eventuella hänvisade boxar
    % annars
        % validera beviset B* som består av radens elementlista med premisserna P fast utan slutsats?

%%%%%%%%%%%%%%%%%%%%%%%%%%%%%%%%%%%%%%%%%%%%%%%%%%%%%%%%%%%%%%%%%%%%%%%
%%%%%%%%%%%%%%%%%%%%%%%%%%%%%%%%%%%%%%%%%%%%%%%%%%%%%%%%%%%%%%%%%%%%%%%
%%%%%%%%%%%%%%%%%%%%%%%%%%%%%%%%%%%%%%%%%%%%%%%%%%%%%%%%%%%%%%%%%%%%%%%

% givet en lista premisser P, en slutsats C samt ett bevis B
% där ett bevis utgörs av en lista element E_0, E_1, ... E_n, där varje element
% antingen är en trippel (radnummer, påstående, motivering) eller
% en lista på ytterligare element (en box)

% att validera ett bevis B

    % utgå från att samtliga radnummer N_n tillhörande E_n \in B initialt är ''stängda''
    % markera samtliga radnummer på lägsta nivån i B som öppna
    % validera det första elementet E_0 \in B

% att validera ett element E_n
    % om E_n inte är en trippel
        % om det första elementet E_m \in E_n inte är en trippel eller inte börjar med motiveringen ''antagande''
            % avbryt och misslyckas
        % annars
            % validera första elementet E_m \in E_n
    % annars
        % välj fall ur motiveringen Mv(E_n)
        % premiss:
            % om inte St(E_n) \in P
                % avbryt och misslyckas
        
    
%%%%%%%%%%%%%%%%%%%%%%%%%%%%%%%%%%%%%%%%%%%%%%%%%%%%%%%%%%%%%%%%%%%%%%%
%%%%%%%%%%%%%%%%%%%%%%%%%%%%%%%%%%%%%%%%%%%%%%%%%%%%%%%%%%%%%%%%%%%%%%%
%%%%%%%%%%%%%%%%%%%%%%%%%%%%%%%%%%%%%%%%%%%%%%%%%%%%%%%%%%%%%%%%%%%%%%%

\subsection{Algoritmens boxhantering}
\noindent
Som ses i algoritmens formella beskrivning ovan hanteras boxar genom att programmet
kan märka vissa radnummer som öppna eller stängda. När en box
påträffas i ett bevisträd (eller ett boxträd) markeras raderna
på den första nivån i trädet som öppna och valideras enligt
precis samma algoritm som elementen i huvudbeviset. När boxen
stängts stängs även raderna. Vid kontroll av bevisregler
görs olika kontroller för hänvisningar till boxar, vilket
bland annat omfattar att kontrollera så samtliga rader i
hänvisade boxar är stängda.
\vfill

\subsection{Predikatdokumentation}
\begin{tabular}{r l}
\textbf{Predikat} & \textbf{Beskrivning}\\
%natrules.pl
% \raisebox{1cm}{\texttt{andel1/2}} & \pbox{20cm}{\texttt{andel1(?X, ?Y)}\\ \\ Sant om X = and(Y, \_)\\ }\\
\predrow{andel1/2}{andel1(?X, ?Y)}{Verifierar den satslogiska bevisregeln $\land e_1$.\\ Sant om X = and(Y, \_).}
% andel2/2 & andel2(?X, ?Y), sant om X = and(\_, Y) \\
\predrow{andel2/2}{andel2(?X, ?Y)}{Verifierar den satslogiska bevisregeln $\land e_2$.\\ Sant om X = and(\_, Y).}
% andint/3 & andint(?X, ?Y, ?Z), sant om Z = and(X, Y) \\
\predrow{andint/3}{andint(?X, ?Y, ?Z)}{Verifierar den satslogiska bevisregeln $\land i$.\\ Sant om Z = and(X, Y).}
% contel/2 & contel(?X, ?Y), sant om X = cont \\
\predrow{contel/2}{contel(?X, ?Y)}{Verifierar den satslogiska bevisregeln $\bot e$.\\ Sant om X = $\bot$.}
% copy/2 & copy(?X, ?Y), sant om X = Y \\
\predrow{copy/2}{copy(?X, ?Y)}{Verifierar den satslogiska bevisregeln \texttt{copy}.\\ Sant om X = Y.}
% impel/3 & impel(?X, ?Y, ?Z), sant om Y = imp(X, Z) \\
\predrow{impel/3}{impel(?X, ?Y, ?Z)}{Verifierar den satslogiska bevisregeln $\rightarrow e$.\\ Sant om Y = imp(X, Z).}
% impint/3 & impint(?X, ?Y, ?Z), sant om Z = imp(X, Y) \\
\predrow{impint/3}{impint(?X, ?Y, ?Z)}{Verifierar den satslogiska bevisregeln $\rightarrow i$.\\ Sant om Z = imp(X, Y).}
% lem/1 & lem(?X), sant om X = or(Y, neg(Y)) \\
\predrow{lem/1}{lem(?X)}{Verifierar den satslogiska bevisregeln LEM.\\ Sant om X = or(Y, $\lnot$ Y) för något Y.}
% mt/3 & mt(?X, ?Y, ?Z), sant om  \\
\predrow{mt/3}{mt(?X, ?Y, ?Z)}{Verifierar den satslogiska bevisregeln MT (modus tollens).\\ Sant om X = imp(A, B), Y = neg(A) och Z = neg(B).}
% negel/3 & ... \\
\predrow{negel/3}{negel(?X, ?Y, ?Z)}{Verifierar den satslogiska bevisregeln $\lnot e$.\\ Sant om Y = neg(X) och Z = $\bot$.}
% negint/3 & ... \\
\predrow{negint/3}{negint(?X, ?Y, ?Z)}{Verifierar den satslogiska bevisregeln $\lnot i$.\\ Sant om Z = neg(X) och Y = $\bot$.}
% negnegel/2 & ... \\
\predrow{negnegel/2}{negnegel(?X, ?Y)}{Verifierar den satslogiska bevisregeln $\lnot \lnot e$.\\ Sant om X = neg(neg(Y)).}
% negnegint/2 & ... \\
\predrow{negnegint/2}{negnegint(?X, ?Y)}{Verifierar den satslogiska bevisregeln $\lnot \lnot i$.\\ Sant om Y = neg(neg(X)).}
% orel/6 & ... \\
\predrow{orel/6}{orel(?X, ?Y, ?Z, ?P, ?Q, ?R)}{Verifierar den satslogiska bevisregeln $\lor e$.\\ Sant om X = or(Y, P) och Z = Q = R.}
\end{tabular}
\newpage

\begin{tabular}{r l}
\textbf{Predikat} & \textbf{Beskrivning}\\
% pbc/3 & ... \\
\predrow{pbc/3}{pbc(?X, ?Y, ?Z)}{Verifierar den satslogiska bevisregeln PBC.\\ Sant om X = neg(Z) och Y = $\bot$.}

%proofs.pl
% all\_open/2 & ... \\
\predrow{open/1}{open(?X)}{Sant om open(X) gäller. Används dynamiskt med \texttt{assert}/\texttt{retract}.}
\predrow{all\_open/2}{all\_open(?X, ?Y)}{Sant om open(X) och open(Y) gäller.}
% all\_open/3 & ... \\
\predrow{all\_open/3}{all\_open(?X, ?Y, ?Z)}{Sant om open(X), open(Y) och open(Z) gäller.}
% none\_open/2 & ... \\
\predrow{none\_open/2}{none\_open(?X, ?Y)}{Sant om varken open(X) eller open(Y) gäller.}
% none\_open/3 & ... \\
\predrow{none\_open/3}{none\_open(?X, ?Y, ?Z)}{Sant om varken open(X), open(Y) eller open(Z) gäller.}
% box\_mark\_open/1 & ... \\
\predrow{box\_mark\_open/1}{box\_mark\_open(+Box)}{Markerar samtliga radnummer i alla tripplar på lägsta nivån i\\ beviset/boxen Box som öppna (\texttt{open}) genom anrop till \texttt{assert}.\\ Alltid sant.}
% box\_mark\_closed/1 & ... \\
\predrow{box\_mark\_closed/1}{box\_mark\_closed(+Box)}{Markerar samtliga radnummer i alla tripplar på lägsta nivån i\\ beviset/boxen Box som stängda genom anrop till \texttt{retract}.\\ Alltid sant.}
% maxbox/2 & ... \\
\predrow{maxbox/2}{maxbox(+List, -Result)}{Plockar fram det högsta ''box-id'' i en lista beräknad av\\ \texttt{lines\_collect\_info/4}.\\ Sant då \texttt{Result} är detta högsta ''box-id''.}
% lines\_collect\_info/2 & ... \\
\predrow{lines\_collect\_info/2}{lines\_collect\_info(+Proof, -Result)}{Går igenom ett bevisträd och beräknar utifrån samtliga tripplar\\ en flat lista på formen \texttt{[[radnummer, box-id, radnummer-i-box], ...]}.\\ Sant då \texttt{Result} är denna lista.}
% lines\_collect\_info/5 & ... \\
\predrow{lines\_collect\_info/5}{-}{Hjälppredikat för \texttt{lines\_collect\_info/2}.}
% line\_in\_box/3 & ... \\
\predrow{line\_in\_box/3}{line\_in\_box(+Proof, +LineNo, ?Box)}{Söker reda på vilken box som radnummret \texttt{LineNo} tillhör. Sant då\\ \texttt{Box} är detta ''box-id''.}
% linenumbers/2 & ... \\
\predrow{linenumbers/2}{linenumbers(+Proof, -List)}{Skapar en lista på samtliga radnummer som förekommer i \texttt{Proof}.\\ Sant då \texttt{List} är denna lista.}
% increasing/1 & ... \\
\predrow{increasing/1}{increasing(+List)}{Sant om \texttt{List} innehåller tal ordnade i icke-strikt stigande ordning.}
% lines\_find\_info/3 & ... \\
\predrow{lines\_find\_info/3}{-}{Hjälppredikat för \texttt{line\_get\_info/3}.}
% line\_get\_info/3 & ... \\
\predrow{line\_get\_info/3}{line\_get\_info(+Proof, +LineNo, -Result)}{Hämtar trippeln för ett givet radnummer ur ett bevisträd. Sant då\\ \texttt{Result} är denna trippel.}
\end{tabular}
\newpage

\begin{tabular}{r l}
\textbf{Predikat} & \textbf{Beskrivning}\\
% box\_max\_line/3 & ... \\
\predrow{box\_max\_line/3}{box\_max\_line(+Proof, +BoxId, -Result)}{Finner det högsta radnummer (globalt i beviset) som tillhör\\ den box som har det givna ''box-id'' \texttt{BoxId}. Detta ''box-id''\\ beräknas på samma sätt som hos \texttt{lines\_collect\_info/2}.\\ Sant när \texttt{Result} är detta radnummer.}
% box\_max\_line/4 & ... \\
\predrow{box\_max\_line/4}{-}{Hjälppredikat för \texttt{box\_max\_line/3}.}
% box\_min\_line/3 & ... \\
\predrow{box\_min\_line/3}{box\_min\_line(+Proof, +BoxId, -Result)}{Finner det lägsta radnummer (globalt i beviset) som tillhör\\ den box som har det givna ''box-id'' \texttt{BoxId}. Detta ''box-id''\\ beräknas på samma sätt som hos \texttt{lines\_collect\_info/2}.\\ Sant när \texttt{Result} är detta radnummer.}
% box\_min\_line/4 & ... \\
\predrow{box\_min\_line/4}{-}{Hjälppredikat för \texttt{box\_min\_line/3}.}
% line\_opens\_box/2 & ... \\
\predrow{line\_opens\_box/2}{line\_opens\_box(+Proof, ?LineNo)}{Sant om någon box öppnas på rad \texttt{LineNo} i beviset.}
% line\_closes\_box/2 & ... \\
\predrow{line\_closes\_box/2}{line\_closes\_box(+Proof, ?LineNo)}{Sant om någon box stängs på rad \texttt{LineNo} i beviset.}
% lines\_delimit\_box/3 & ... \\
\predrow{lines\_delimit\_box/3}{lines\_delimit\_box(+Proof, ?Line1, ?Line2)}{Sant om någon box öppnas på rad \texttt{Line1} och stängs på rad\\ \texttt{Line2} i beviset. \texttt{Line1} och \texttt{Line2} måste tillhöra samma box\\ och finnas på samma djup.}
% first\_opened\_box\_in\_box/3 & ... \\
\predrow{first\_opened\_box\_in\_box/3}{first\_opened\_box\_in\_box(+Proof, +OwnBox, ?BoxNo)}{Finner ''box-id'' på den första box som öppnas i den box vars\\ ''box-id'' är \texttt{OwnBox}. Sant när \texttt{BoxNo} är detta ''box-id''.}
% most\_recently\_closed\_box/3 & ... \\
\predrow{most\_recently\_closed\_box/3}{most\_recently\_closed\_box(+Proof, +LineNo, ?BoxNo)}{Finner ''box-id'' på den box som senast stängdes sett från ett\\ givet radnummer \texttt{LineNo}. Sant när \texttt{BoxNo} är detta ''box-id''.}
% proof\_nth\_line/3 & ... \\
\predrow{proof\_nth\_line/3}{proof\_nth\_line(+Proof, +N, ?Line)}{Hämtar den trippel i \texttt{Proof} vars radnummer är \texttt{N}.\\ Sant när \texttt{Line} är denna trippel.}
% proof\_nth\_statement/3 & ... \\
\predrow{proof\_nth\_statement/3}{proof\_nth\_statement(+Proof, +N, ?Statement)}{Hämtar påståendet ur trippeln vars radnummer är \texttt{N}.\\ Sant när \texttt{Statement} är detta påstående.}

% labb1.pl
% verify/1 & ... \\
\predrow{verify/1}{verify(+Filename)}{Läser in premisser, slutsats och bevis från filen \texttt{Filename}.\\ Om beviset är korrekt skrivs ''yes'' ut på standard out,\\ annars skrivs ''no'' ut på standard out. Predikatet i sig är\\ alltid sant.}
% valid\_proof/3 & ... \\
\predrow{validate\_proof/3}{validate\_proof(+Premises, +Conclusion, +Proof)}{Validerar ett bevis enligt tidigare beskrivna algoritm.\\ Sant om beviset är korrekt.}
% validate/3 & ... \\
\predrow{validate/3}{validate(+Proof, +Premises, +Element)}{Validerar ett element enligt tidigare beskrivna algoritm.\\ Sant om algoritmen inte lyckas hitta någon anledning\\ att misslyckas.}
% main/1 & ... \\
\end{tabular}

\newpage
\section{Resultat}
\noindent
Programmet klarade samtliga testfall (valid01-valid21, invalid01-invalid28)
som fanns tillgängliga via kurshemsidan, samt våra egna två testfall (se bilagor).

\section{Bilagor}
\subsection{labb1.pl}
\lstinputlisting[tabsize=4]{labb1.pl}

\newpage
\subsection{proofs.pl}
\lstinputlisting[tabsize=4]{proofs.pl}

\newpage
\subsection{natrules.pl}
\lstinputlisting[tabsize=4]{natrules.pl}

\newpage
\subsection{ourvalid.txt}
\lstinputlisting[tabsize=4]{tests/ourvalid.txt}

\bigskip
\subsection{ourinvalid.txt}
\lstinputlisting[tabsize=4]{tests/ourinvalid.txt}

\end{document}

% \subsection{Algoritm}
% \textbf{Indata}: en lista med premisser $P$, en slutsats $S$ och ett bevis $B$ där $B$ utgörs av en lista där varje element $E_n$ antingen är en trippel på formen (radnummer, påstående, motivering) eller en lista innehållande fler element (en box).

% backus naur tar vi sen
% https://en.wikipedia.org/wiki/Backus%E2%80%93Naur_Form

% Definitioner:
% Ln(E) : radnumret i en trippel
% St(E) : påståendet i en trippel
% Mt(E) : motiveringen i en trippel
% Rr(E) : de radnummer på samma boxnivå som hänvisas till i motivering
% Rb(E) : de radnummer på djupare boxnivå som hänvisas till i motivering

% Att validera ett bevis $B$ givet en lista med premisser $P$ samt en slutsats $S$:
  % Markera samtliga bevisrader på djup 0 som öppna.
  % Validera det första elementet $E_0 \in B$
  % Kontrollera att det sista elementet $E_n \in B$ är en trippel och
  % att dess påstående är lika med $S$.

% Att validera ett element $E_n$ givet ett bevis $B$ samt en lista med premisser $P$:
  % OM $E_n$ är en trippel
    % OM Ln(E_n) är stängd
      % misslyckas
    % OM någon rad \in Rr(E_n) är stängd
      % misslyckas
    % OM någon rad \in Rr(E_n) är större eller lika med Ln(E_n)
      % misslyckas
    % OM någon rad \in Rb(E_n) är öppen
      % misslyckas
    % VÄLJ Mt(E_n) ur
      % “premiss”:
        % OM St(E_n) \notin P
          % misslyckas
      % “copy(k)”:
        % OM St(E_n) inte är lika med St(E_k)
          % misslyckas
      % “negationseliminering(k,m)”:
        % OM St(E_k) inte är lika med icke-St(E_m) ELLER St(E_n) != falsum
          % misslyckas
    % “falsumeliminering(k)”:
       % OM St(E_k) != falsum
         % misslyckas
     % “ickeickeintroduktion(k)”:
        % OM St(E_n) inte är lika med icke-icke-St(E_k)
          % misslyckas
     % “LEM”:
        % OM St(E_n) inte är ett påstående på formen “X eller icke-X”
          % misslyckas
     % “MT(k,m)”:
        % OM St(E_k) inte anger att icke-St(E_n) implicerar icke-St(E_m)
          % misslyckas
     % “implikationseliminering(k,m)”
        % OM St(E_m) inte anger att St(E_k) implicerar St(E_n)
          % misslyckas
     % “ocheleminering1(k)”:
        % OM St(E_k) inte är av formen “St(E_n) ^ X”
          % misslyckas
     % “ocheleminering2(k)”:
        % OM St(E_k) inte är av formen “X ^ St(E_n)”
          % misslyckas
     % “ochintroduktion(k,m)”:
        % OM St(E_n) inte är lika med “St(E_k) ^ St(E_m)”
          % misslyckas


     % “ickeickeeliminering(k)”:
       % OM St(E_k) inte är icke-icke-St(E_n)
     % misslyckas
% “antagande”:
   % OM Ln(E_n) inte öppnar en box
     % misslyckas
% “ellereliminering(k,m,p,q,r)”:
   % OM raderna E_m och E_p inte öppnar respektive stänger en box
     % misslyckas
   % OM raderna E_q och E_r inte öppnar respektive stänger en box
     % misslyckas
   % OM St(E_k) inte är lika med “St(E_m) v St(E_q)”
     % misslyckas
   % OM St(E_p) inte är lika med St(E_r) ELLER St(E_n) inte är lika med St(E_p)
     % misslyckas
% “pbc(k,m)”:
   % OM E_k och E_m inte står som öppnande respektive stängande rad för senast stängda box:
     % misslyckas
   % OM St(E_k) inte är lika med icke-St(E_n) ELLER St(E_m) inte är lika med falsum
     % misslyckas
     % “implikationsintroduktion(k,m)”:
        % OM E_k och E_m inte står som öppnande respektive stängande rad för senast stängda box
          % misslyckas
        % OM St(E_n) inte är lika med “St(E_k) -> St(E_m)”
          % misslyckas
     % “negationsintroduktion(k,m)”:
        % OM E_k och E_m inte står som öppnande respektive stängande rad för senast stängda box
          % misslyckas
        % OM St(E_m) inte är lika med falsum ELLER St(E_n) inte är lika med icke-St(E_k)
          % misslyckas
     % ANNARS:
        % OM E_n inte är en lista vars första element är en trippel med
% motiveringen “antagande”
          % misslyckas
        % Markera samtliga rader på djup 0 i E_n som öppna
        % Validera [E_n]_0
        % Markera samtliga rader på djup 0 i E_n som stängda


 % ANNARS 
   % Validera $E_{n+1}$
